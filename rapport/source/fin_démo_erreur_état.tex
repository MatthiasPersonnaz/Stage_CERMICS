
et donc avec \ref{reordonnement} on conclut: puisque $\vect{r}$ est la solution d'un système linéaire dont $\vect{0}$ est solution\footnote{Correspondrait au cas où l'erreur sur $\Delta \Psi^s$ et l'erreur $e$ sur l'énergie sont nulles.}, on a l'inégalité de conditionnement
\begin{equation}
    \boxed{\|\vect{r}\|_2 \le \text{cond} \left(-\frac{\hbar^2}{2m}\Lambda +V - E \right)   \times \left( \frac{\hbar^2\textcolor{blue}{g}}{2m} + eD \right)}
\end{equation}

La grandeur $e$ (i.e. erreur sur l'énergie faite en discrétisant) peut être contrôlée typiquement avec l'inégalité de \textsc{Kato-Temple} par exemple.


\paragraph{Troisième étape: expression avec les paramètres modifiables}
En réécrivant tout en remplaçant $\textcolor{blue}{g}$ par son expression:
\begin{equation}
    \|\vect{r}\|_2 \le \text{cond} \left(-\frac{\hbar^2}{2m}\Lambda +V - E \right)   \times \left( \frac{\hbar^2}{2m}\left( \frac{Nh^2S}{12} + \frac{2\gamma\sqrt{N} }{h^2}  \right) + eD \right)
\end{equation}
et en supposant ici $x_\text{max} = -x_\text{min}$, on a $h = \frac{x_\text{max}-x_\text{min}}{N} = \frac{2x_\text{max}}{N}$ et rappelons qu'alors $\gamma = \mathcal{O}(\exp(-x_\text{max}^2))$:
\begin{equation}
    \|\vect{r}\|_2 \le \text{cond} \left(-\frac{\hbar^2}{2m}\Lambda +V - E \right)   \times \frac{\hbar^2}{2m} \left( \frac{x_\text{max}^2 S}{3N} + \frac{8N^{5/2} \mathcal{O}(\exp(-x_\text{max}^2))}{x_\text{max}^2} + eD \right)
\end{equation}


Pour contrôler l'erreur, cela implique donc de jouer sur les paramètres $x_\text{max}$ et $N$ que l'on peut choisir indépendamment. Le conditionnement de la matrice $\text{cond} \left(-\frac{\hbar^2}{2m}\Lambda +V - E \right)$ n'est pas évident non plus. Nous savons simplement que $\text{cond} \left(\Lambda \right) \sim   \frac{4}{\pi^2h^2}$, et que $V$ est diagonale, ainsi $\text{cond}(V) = \frac{V_\text{max}}{V_\text{min}}$. Numériquement, on peut dans un cas particulier estimer ce conditionnement (sur GPU, etc).

On a $\mathcal{O}\left( \left| \frac{\max_{[x_\text{min},x_\text{max}] \times [y_\text{min},y_\text{max}]} E-\mathcal{V}}{\min_{[x_\text{min},x_\text{max}] \times [y_\text{min},y_\text{max}]} E-\mathcal{V}}\right|\right)$ 
et puisque $V \rightarrow +\infty$, $ \left| \frac{\max E-V}{\min E-V}\right|  = \mathcal{O}(V_\text{max})$ où $V_\text{max} = \mathcal{V}(x_\text{max})$.


\begin{equation}
    \boxed{\|\vect{r}\|_2 =  \left[\mathcal{O} \left( \frac{N^2}{x_\text{max}^2} \right) + \mathcal{O}(V_\text{max})  \right]     \left[ \mathcal{O} \left( \frac{x_\text{max}^2}{N} \right) + \mathcal{O} \left( \frac{N^{5/2} \exp(-x_\text{max}^2)}{x_\text{max}^2} \right) + \mathcal{O}(e) \right]}
\end{equation}

À première vue, il n'y a donc pas de moyen simple de faire converger l'erreur vers 0 à coup sûr avec cette majoration, en gardant toujours à l'esprit qu'elle perd beaucoup d'information. Globalement, faire grandir $x_\text{max}$ permet très rapidement d'annuler l'erreur faite sur le hamiltonien, mais  Cette analyse est utile asymptotiquement, et montre qu'il ne faut pas faire grandir $N$ démesurément à largeur de grille fixée car cela augmente l'erreur.  En revanche, élargir la grille est  une bonne idée pour approcher la fonction d'onde et n'augmente pas sensiblement l'erreur (conformément à l'intuition, on commet moins d'erreur en \og contraignant \fg{} à tort la particule ainsi).


On peut revenir à l'erreur en valeur absolue mesurée par $\|\cdot \|_\infty$, vérifiant $\| \vect{r} \|_\infty \le \| \vect{r} \|_2$, et espérer raisonnablement qu'elle soit  inférieure au moins d'un ordre de grandeur puisque la norme 2 est une somme sur tous les points de la grille.

Il est compliqué néanmoins de conclure précisément sur la majoration car certains paramètres existent (grâce au théorème \ref{thm_majoration_fct_onde} notamment) mais ne sont pas évidents à exhiber.

Il est à noter que mes majorations perdent beaucoup d'information, et ne sont certainement pas optimales, en raison du caractère très localisé dans l'espace de la fonction d'onde recherchée, et les inégalités d'analyse numérique utilisées ne prennent pas ceci en compte.